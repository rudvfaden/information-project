\documentclass[10pt,a4paper]{article} % KOMA-Script article scrartcl

%%--------------------------------------------------------------------------
%% Colors and hypersetup
%%--------------------------------------------------------------------------
\hypersetup{
	colorlinks = true,      					%	Color links instead of ugly boxes
	urlcolor   = webbrown,  					%	Colors for external hyperlinks
	linkcolor  = blue,      					%	Colors of internal links
	citecolor  = webgreen,						%	Colors of citations
	pdfauthor = {Rud Faden: rudfaden@gmail.com},
	pdfcreator = {LaTeXing}
}

%%--------------------------------------------------------------------------
%% Theorem & Definitions
%%--------------------------------------------------------------------------

% Theorem Styles
\newtheorem{thm}{Theorem}[section]
\newtheorem{lemma}[thm]{Lemma}
\newtheorem{prop}[thm]{Proposition}
\newtheorem{corollary}[thm]{Corollary}
\newtheorem{assumption}[thm]{Assumption}
% Definition Styles
\theoremstyle{definition}
\newtheorem{defn}{Definition}[section]
\newtheorem{example}{Example}[section]
\theoremstyle{remark}
\newtheorem{remark}{Remark}


%%--------------------------------------------------------------------------
%% User specified LaTeX commands.
%%--------------------------------------------------------------------------

\global\long\def\argmax{\operatornamewithlimits{arg\, max}} % make the \argmax command available
\let\marginpar\oldmarginpar% fix for todonotes margin notes
\makeatletter
\DeclareRobustCommand{\pder}[1]{%
  \@ifnextchar\bgroup{\@pder{#1}}{\@pder{}{#1}}}
\newcommand{\@pder}[2]{\frac{\partial#1}{\partial#2}}
\makeatother
%%--------------------------------------------------------------------------
%% Misc
%%--------------------------------------------------------------------------

\setkeys{Gin}{width=1\textwidth} % make figures as wide as the margins
%solves wird tikz error
\makeatletter
\global\let\tikz@ensure@dollar@catcode=\relax
\makeatother
%%--------------------------------------------------------------------------
%% title and author
%%--------------------------------------------------------------------------

\title{\rmfamily\normalfont\spacedallcaps{Physician Information Acquisition In a Dynamic Setting} \\ {\large Knowledge Accumulation and Continuity of Care}}
\author{\spacedlowsmallcaps{Rud Faden}}
\date{} % no date

\begin{document}

\maketitle

%%--------------------------------------------------------------------------
%% Abstract
%%--------------------------------------------------------------------------

\begin{abstract}
In this project, I examine provider and patient demand for information in a dynamic model where the diagnostic precision is assumed to be related to physician effort, and effort is non-contractible. In each period where the patient and physician interact, the physician gathers information about the patient, and the diagnostic precision is increased. Therefore, optimal physician effort decreases as the physician and patient tie increases. As the physician is unobserved, the insurer compensates the physician by the average effort in the physician population and physician will not provide an optimal level of diagnostic precision in the in the first encounters with a new patient. Therefore the switching cost of the patient increases as the tie with the physician lengthens. This model explains (i) why the cost is negatively related with patient, physician ties and (ii) also introduces the concept of an ``information trap'', where competition is deceasing in the patient physician tie as switching cost increases. Increases.
\end{abstract}

%%--------------------------------------------------------------------------
%% Section: The Patients Utility}
%%--------------------------------------------------------------------------

\section{The Patients Utility}

Following \citet{Rochaix1989}, the patient has a utility function
\begin{eqnarray*}
U & = & u(t,s):\, T\times S\rightarrow\mathbb{R}
\end{eqnarray*}
 where $t$ is treatment, $s$ is disease variable classified by its severity of illness, where both $t$ and $s$ traverse a real line. It is assumed that the utility function has both increasing and decreasing parts to capture the negative effects of both under and over treatment. It is further assumed that the dis-utility of moving away from the optimum is increasing in $s$, such that the patient is more \emph{risk-sensitive} for higher values of $s$.

%========== DEFINITION: Risk-sensitivity ==========%
\begin{defn}
\label{def:risk-sensitivity} 
Assuming that the decision problem has a unique solution $t^{*}(s)$, and for $s'>s$, $u(t^{*}(s),s)=u(t^{*}(s'),s')$. Then for $t^{*}(s)>t_{1}$, $t^{*}(s')>t_{2}$, $t_{2}\ge t_{1}$ and $t^{*}(s')-t_{2}=t^{*}(s)-t_{1}=\Delta t$, the \emph{risk-sensitivity} is monotonically increasing in $s$, if
\begin{equation}
0\le u(t^{*}(s),s)-u(t_{1},s)\le u(t^{*}(s'),s')-u(t_{2},s')\label{eq:risk-sensitivity}
\end{equation}
for all $s$ and $t$, and I write that $u(t_{1},s)\preceq u(t_{2},s')$.
\end{defn}
%==================================================%

The notion of \emph{risk-sensitivity} is illustrated in figure\ref{fig:The-patients-utility}.

%========== THEOREM: Single crossing property ==========%
\begin{thm}
\label{thm:single-crossing} 
If the risk sensitivity is increasing in $s$, then for $s'>s$ $u(t,s)$ has a single crossing property in $(s,t)$
\end{thm}
%==================================================%

%========== PROF - crossing property ==========%
\begin{proof}
Let $t'=t^{*}(s')$ and $t=t^{*}(s)$. Given definition\ref{def:risk-sensitivity}, it is clear that $t'\ge t_{2}>t\ge t_{1}$. Therefore I can rewrite
\eqref{eq:risk-sensitivity} as

\begin{equation}
u(t',s')-u(t_{2},s')\ge u(t,s)-u(t_{1},s)\ge0\label{eq:increasing-differences}
\end{equation}

\eqref{eq:increasing-differences} clearly have increasing differences in $s$ and thereby satisfies the single crossing property.
\end{proof}
%==================================================%

The intuition behind theorem\ref{thm:single-crossing}, is that the marginal change $u(t',\cdot)-u(t,\cdot)$ is larger, when $s$ is larger. When $u(t,s)$ is differentiable and concave as in figure \ref{fig:The-patients-utility}, one might note $\partial u(t,s)/\partial t\ge0$ for $t\le t^{*}(s)$ and that for $\partial u(t,s)/\partial t\le0$ for $t\ge t^{*}(s)$. Thereby $\partial u(t,s')/\partial t>\partial u(t,s)/\partial t$ for $t\le t^{*}(s)$ and $\partial u(t,s')/\partial t<\partial u(t,s)/\partial t$ for $t\ge t^{*}(s)$. If $u(t,s)$ is convex, then one might note that $\partial u(t,s')/\partial t$ crosses $\partial u(t,s)/\partial t$ at most once and from below.

%========== Begin figure: Utility and single-crossing ==========%
\begin{figure}
	\subfloat[Utility]{
		\resizebox{0.45\textwidth}{!}{%
		\includegraphics{../fig/patient-utility.pdf}
	}
}
	\subfloat[Single Crossing]{
		\resizebox{0.45\textwidth}{!}{%
		\includegraphics{../fig/single-crossing-property.pdf}
	}
}

\caption{\label{fig:The-patients-utility}The patients utility functions. If for the same change in $t$ the loss in utility is smaller for $s$ than for $s'$, when $s<s'$, then I says that risk sensitivity isincreasing in $s$.}
\end{figure}
%========== End figure: Utility ==========%

%========== Begin proposition: Single-crossing property ==========%
\begin{prop}
Given that $u(t',s)-u(t,s)$ has a single crossing property in $s$ and that both $S$ and $T$ are well ordered sets, then

\[
t^{*}(s)=\argmax_{s\in S}u(t,s)
\]

is increasing in $s$.\footnote{It should be noted that the assumption of quasisupermodularity is not needed as the choice space is well ordered (e.i.\ a chain). %
}
\end{prop}
%========== End proposition: Single-crossing property ==========%

%========== Begin prof: Single-crossing ==========%
\begin{proof}
As both $T$ and $S$ are real lines, then it follows trivially that they are well ordered. For the rest of the proof see \citet{Milgrom1994}
\end{proof}
%========== End prof: Single-crossing ==========%

%========== Begin example: Single-crossing ==========%
\begin{example}
A function with the properties defined in theorem\ref{thm:single-crossing} and definition\ref{eq:risk-sensitivity} and have the form as in figure\ref{fig:The-patients-utility} is

\begin{equation}
u(t,s)=c-s{(s-t)}^{2}\label{eq:utility-example}
\end{equation}

where $t,s\in\mathbb{R}^{+}$. Assuming that\eqref{eq:utility-example} is continuous and twice differential in $t,s$, the derivative $\partial^{2}u(t,s)\big/\partial t\partial s=2s^{2}\ge0$ and~\eqref{eq:utility-example} has increasing differences and thereby also a single crossing property in $(t,s)$.
\end{example}
%========== End example: Single-crossing ==========%

%%--------------------------------------------------------------------------
%% Uncertainty with perfect agency
%%--------------------------------------------------------------------------

\section{Uncertainty with perfect agency}

In reality however, $s$ is never observed. The level of severity for the patients is a random variable represented by $S$, characterized by a subjective cdf. $F(s)$, with density $f(s)$, where $s$ is a realization of $S$. The expected value of choosing an admissible treatment intensity$t$ is given by

\begin{equation}
u(t,S)=E[u(t,s)]=\max_{t}\int_{S}u(t,s)dF(s)\label{eq:expected-utility-prior}
\end{equation}

It is however possible to acquire costly information about $s$ through medical diagnostics and physician effort. However, for two experiment $X,Y$ on $S$, it is not a priori certain that one experiment $X$ is necessary more \emph{informative} about $s$ than the experiment $Y$, where \emph{informative} is to be understood in the way the posterior decision induced by the experiment $X$ insures greater expected utility than the decision induced by the experiment $Y$. Therefore we must introduce an order of information.

%%--------------------------------------------------------------------------
%% Section: Information ordering
%%--------------------------------------------------------------------------

\subsection{Information ordering}

%========== Begin definition: affiliation ==========%
\begin{defn}
\label{def:afflilation} (\citet{Milgrom1982}) 
For a family of densityfunctions, let $x\lor s$ denote the component wise maximum and $x\land s$the component wise minimum. Then $x$ and $s$ are affiliated if,for all $s$ and $x$
\[
f(s\lor x)f(s\land x)\ge f(s)f(x)
\]
\end{defn}
%========== End definition: affiliation ==========%

Affiliation of two random variables are equivalent to the monotone likelihood ration property, and the intuition behind definition~\ref{def:afflilation} is that higher signal realization of $x$ makes the probability that $s$ is large, higher. Similarly small signal realization of $x$ makes the probability of a small $s$ more likely.

%========== Begin definition: Accuracy ==========%
\begin{defn}\label{def:accuracy}
(\citet{Persico2000}) Given two signals
(experiments) $X^{\eta}$ and $X^{\eta'}$, $X^{\eta'}$ is more accurate than $X^{\eta}$ if

\begin{equation}
T(x)=F^{\eta'^{-1}}(F^{\eta}(x\mid s)\mid s)\label{eq:acuracy tranformation}
\end{equation}

is non decreasing in $s$ for all $x$.

A family of signals $\left\{ X^{\eta}\right\} _{\eta\in E}$ is accuracy ordered (A-ordered) if a signal with higher index is more accurate than a signal with lower index.
\end{defn}
%========== End definition: Accuracy ==========%

To understand the concept of accuracy, it can be noted that
\[
T(X^{\eta}\mid s)\sim X^{\eta'}\mid s
\]

Thus a more accurate signal can be obtained from a less accurate signal, by the transformation $T(X)$. An example is given in example

%========== Begin example: T-transformation 1 ==========%
\begin{example}
Let  $\eta\in[0,\infty]$and let $S$ be distributed according to any CDF and let $X_{\eta}\sim\mathbb{U}(s-1/\eta,s+1/\eta)$. Then for $\eta'>\eta$

\[
\frac{\eta'}{2}=\frac{\eta}{2}\left(T^{-1}(x)\right)'\Leftrightarrow T(x)=\frac{\eta}{\eta'}(x-s)+s
\]

Thus, $T(x)$ transforms $X^{\eta}$ into $X^{\eta'}$ \citep{Persico1996}.
\end{example}
An illustrative example can also be given by applying $\eqref{eq:acuracy tranformation}$ to hypothesis testing.

%========== Begin Example: T-transformation 2==========%
\begin{example}
Consider the case where $s$ can take two values $s_{1}<s_{2}$. Let $X^{\eta}$ be a information structure affiliated with $S$. The optimal test based on $X^{\eta}$ is given by the rejection region $X^{\eta}>x^{*}$, such that $s_{1}$ is rejected in favor of $s_{2}$ when $X^{\eta}>x^{*}$. The probability of a type I error is then $\mathbb{P}(X^{\eta}\le x^{*})=F(x^{*}\mid s_{2})$ and the probability of a type II error is $1-F(x^{*}\mid s_{1})$. Now given that $X^{\eta'}$ is more accurate than $X^{\eta}$ is is possible to design a test with the same probability of type I error, by choosing $x^{**}$ such that $F(x^{**}\mid s_{2})=F(x^{*}\mid s_{2})$ (e.i. \ accept $s_{2}$ if $X^{\eta}\ge x^{**}$). However, since $X^{\eta'}$ is more accurate than $X^{\eta}$, then $x^{**}\ge F^{\eta'^{-1}}(F^{\eta}(x\mid s_{1})\mid s_{1})$. As $x^{**}$ lies on or, to the right of $x^{*}$ then the test based on $X^{\eta'}$ is a least as powerful as the test based on $X^{\eta}$
\citep{Lehmann1988,Persico2000}.

\begin{figure}
\includegraphics[width=\textwidth]{../fig/t-transformation.pdf}
\protect\caption{The $T(x)$ transformation}
\end{figure}

\end{example}
%========== End example: T-transformation ==========%

%%--------------------------------------------------------------------------
%% Subsection: Demand for information
%%--------------------------------------------------------------------------

\subsection{Demand for information}

Given that we now know, when a test can be considered more informative than another, I can know turn to the problem of informativeness.

%========== Begin theorem: ilnformative ==========%
\begin{thm}
\label{thm:informative}
(\citet{Persico2000}) 
Suppose that $X^{\eta}$ and $X^{\eta'}$ are affiliated with $S$ and that $\left\{ X^{\eta}\right\} _{\eta\in E}$ is A-ordered, such that $X^{\eta'}$ is more accurate than $X^{\eta}$. Then for all utility functions with a single crossing properlty $X^{\eta'}$is more informative than $X^{\eta}$.
\end{thm}
%========== End theorem: informative ==========%

%========== Begin prof: informative ==========%
\begin{proof}
See \citet{Lehmann1988} section 4 and \citet{Karlin1956} Lemma 3--4 and theorem 1.
\end{proof}
%========== End prof: informative ==========%

It follows directly from Theorem~\ref{thm:informative} that when $X^{\eta'}$ is more informative than $X^{\eta}$, then

\[
\int_{X}\int_{S}u(t,s)dG^{\eta'}(s\mid x)dF(x)\ge\int_{X}\int_{S}u(t,s)dG^{\eta}(s\mid x)dF(x)
\]

%%--------------------------------------------------------------------------
%% Section: The physician problem
%%--------------------------------------------------------------------------

\section{The physician problem}

In this section we will look at the physicans problem. Like in \citet{Allard2009} and \citet{Weiss1996}

\section{Dynamic setting}

\printbibliography
\end{document}
