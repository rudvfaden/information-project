%% LyX 2.1.0 created this file.  For more info, see http://www.lyx.org/.
%% Do not edit unless you really know what you are doing.
\documentclass[english,paper=a4,fontsize=10pt]{scrartcl}
\usepackage[T1]{fontenc}
\usepackage[latin9]{inputenc}
\usepackage{color}
\usepackage{babel}
\usepackage{amsthm}
\usepackage{amsmath}
\usepackage{amssymb}
\usepackage{esint}
\usepackage[unicode=true,
 bookmarks=true,bookmarksnumbered=true,bookmarksopen=false,
 breaklinks=false,pdfborder={0 0 0},backref=false,colorlinks=true]
 {hyperref}
\hypersetup{
 urlcolor=webbrown,linkcolor=Blue,citecolor=webgreen}

\makeatletter
%%%%%%%%%%%%%%%%%%%%%%%%%%%%%% Textclass specific LaTeX commands.
\RequirePackage{todonotes}
\theoremstyle{plain}
\newtheorem{thm}{\protect\theoremname}
  \theoremstyle{definition}
  \newtheorem{defn}[thm]{\protect\definitionname}

%%%%%%%%%%%%%%%%%%%%%%%%%%%%%% User specified LaTeX commands.
% article example for classicthesis.sty
 % KOMA-Script article 
\usepackage{lipsum}%\usepackage{../classicthesis-ldpkg}
\usepackage[nochapters]{classicthesis}% nochapters
\let\oldtitle\title
\renewcommand{\title}[1]{\oldtitle{\rmfamily\normalfont\spacedallcaps{#1}}}
\let\oldauthor\author
\renewcommand{\author}[1]{\oldauthor{\spacedlowsmallcaps{#1}}}
\date{}

\definecolor{Blue}{cmyk}{1, 0.878, 0, 0}
\usepackage{tikz}
\usepackage{amssymb}

\makeatother

  \providecommand{\definitionname}{Definition}
\providecommand{\theoremname}{Theorem}

\begin{document}

\title{Physician Information In a Dynamic Setting}


\author{Rud Faden}

\maketitle
\tableofcontents{}


\section{The Patient}

The patient has a utility function 
\begin{eqnarray*}
U & = & u(h;\pi(d\mid s))\\
h & = & h(t,d)
\end{eqnarray*}


where $h$ is health, $t$ is treatment, $d$ is a random disease
variable which is unobserved, but has a known distribution $\pi(d)$,
$s$ is a signal about $d$ and it is assumed that $\pi(s\mid d)$
is known to the physician, but not to the patient, such that the physician
can infer $\pi(d\mid s)$; the probability distribution of $d$ given
$s$.

It is assumed that for each $s'>s$ $\pi(d\mid s')$ first order stochastically
dominates $\pi(d\mid s)$ 
\begin{defn}
\label{def:-first-order-sthocastic-domination}$\pi(d\mid s')$ first
order stochastically dominates $\pi(d\mid s)$ if and only if 
\[
\pi(d\mid s')\le\pi(d\mid s)
\]


and for every non-decreasing function $u:\mathbb{R}\rightarrow\mathbb{R}$
it holds that 
\begin{equation}
Eu(h;\pi(d\mid s'))\ge Eu(h;\pi(d\mid s))\label{eq:D1-fosd}
\end{equation}

\end{defn}

\subsection{Static setting under full information}

To derive the optimal behavior of the patient, I start by analyzing
the patient optimal treatment decision $d$, where the patient has
the same information as the physician. for simplicity, I assume now
that the only relevant costs are $s$ produced at a cost 
\[
c=c(s)
\]


where $c'>0$ and $c''>0$. Given this, a rational agent would acquire
a volume of $s$ given by 
\[
Eu(h-c(s);\pi(d\mid s))=Eu(h)\pi(d)
\]


That is, the expected gain in utility from a decrease in uncertainty,
must equal the loss in utility from the cost of obtaining signals
$s.$ 


\subsection{Dynamic analysis under full information}

Now given that in each period the patient observes signal $s$ and
that over time, this information is aggregated into $s_{i}$, such
that 
\[
s_{i}=s+s_{i-1}+\cdots s_{1}
\]


where $s$ indicates the amount of signals acquired in the present
period, Then I can write the patients utility as 
\[
Eu(h_{i}-c_{i}(s);\pi(d_{i}\mid s_{i}))
\]


Then clearly as $\pi(d\mid s_{i})\le\pi(d\mid s_{i-1})$ the amount
of information optimally acquired acquired at time $i$ is less that
the amount of information optimally acquired at time $i-1$.
\begin{proof}
Assume that the same amount for information is acquired in each period,
such that $\bar{c}(s)=c_{i}(s)=c_{i-1}(s)$. Then given definition
\ref{def:-first-order-sthocastic-domination} it must be true that
\begin{equation}
Eu(h_{i}-\bar{c}(s);\pi(d_{i}\mid s_{i}))\ge Eu(h_{i-1}-\bar{c}(s);\pi(d_{i-1}\mid s_{i-1}))\label{eq:proff-signal}
\end{equation}


However if \eqref{eq:proff-signal} holds, then it must also be true
that for 
\[
Eu(h_{i}-c_{i}(s);\pi(d_{i}\mid s_{i}))=Eu(h_{i-1}-c_{i-1}(s);\pi(d_{i-1}\mid s_{i-1}))
\]


$c_{i}(s)\le c_{i-1}(s)\Leftrightarrow s_{i}\ge s_{i-1}$ or that
$s$ at time $i$ is less than $s$ at time $i-1$.
\end{proof}

\section{The insurance}

It is assumed that the cost of providing treatment $t$, $C_{t}$
can be perfectly observed by the insurance, such that the price $P_{t}=C_{t}$.
However, the cost of acquiring signals $s$ (e.i. diagnosing the patient)
$C(s)$ is unobserved. Therefore the payment for $s$ is based on
the average cost of diagnosing $t$ over all time periods
\[
\delta^{*}=\int_{i}\int_{s}C(s)d\pi(s\mid d)di
\]



\section{The physician}

The physician has a net income of 
\[
y(t,s)=P_{t}+\delta-C_{t}-C(s)+\lambda Eu
\]


where $P_{t}$ is the revenue generated for treatment $t$, $\delta$
is the per--patient, per--period fee and $C_{t})$ is the cost of
treatment $t$ and $C(s)$ is the cost signals $s$. It is assumed
that $C(s)$ is convex in $s$ ($C_{s}'>0,C_{s}''>0$). This captures
the idea that information acquisition is more costly, the more information
is needed, due to fact that I assume that the physician acquires the
least costly information first. The last term is the weight the physician
puts on patient utility. 


\subsection{Static analyzes}

Like the patient I assume that the physician knows $\pi(d)$, $\pi(s\mid d)$
and $\pi(s)$ , but that $d$ is unobserved. 

As it is assumed that the insurance does not know $C(s)$, but can
only infer the average cost of $C(s)$ , $\delta^{*}$. However, after
$\delta^{*}$ is determined, each physician my choose a different
level of $s$, depending on his level of altruism $\lambda$.

However, as the physician may acquire more or less signals than the
average, the physician realized payment is 
\[
y(s,t)=\delta^{*}-C(s)+\lambda Eu,\text{ as }P_{t}=C_{t}
\]


Assuming that the physician will only treat the patient if he makes
a profit, I define the physicians participation constraint by 
\[
\delta^{*}+\lambda Eu\ge C(s)
\]


As the only the patients expected utility and the physicians cost
change with $s$, we must have that 
\begin{align}
\frac{\partial}{\partial s}(\delta^{*}-C(s)) & =\lambda\frac{\partial Eu}{\partial s}\ \text{ and }\label{eq:max-physician}\\
\delta^{*}+\lambda Eu & \ge C(s)\label{eq:eq:constraint-physician}
\end{align}


Given \eqref{eq:eq:constraint-physician}, we know that no physician
will ever acquire more signals than
\[
\delta^{*}+\lambda Eu=C(s)
\]


Solving for $\lambda$, I find that 
\[
\lambda=\frac{C(s)-\delta^{*}}{Eu}
\]


Substituting into \eqref{eq:max-physician} I get 
\begin{equation}
\frac{1}{Eu}\frac{\partial Eu}{\partial s}=-\frac{1}{(\delta^{*}-C(s))}\frac{\partial}{\partial s}(\delta^{*}-C(s))\label{eq:max-signal}
\end{equation}


It is clear from \eqref{eq:max-signal} that the no physician will
provide signals above the point where the semi elastic change in expected
utility is equal to the loss in profit. 

\todo[inline]{Her vil jeg gerne vise at l�gen at l�gen i visse tilf�lde vil underbhandle,
dat $\delta^{*}$er baseret p� gennemsnittet af omkostningerne og
ikke de faktiske omkostninger. S� kun persone p� venste side af fordelingen
vil f� optimal behandling. Har du nogen gode ideer?

Her ville jeg s� g� videre til et dynamisk setting og vise at hvis
patienten ser l�gen mange gange, s� vil alle f� den korrekte behandling
og l�gen ville blive betalt mere en h�jest n�dvendigt. Samtidigt er
omkostningerne ved at skifte h�jere for patienten, da patientens omkostninger
ogs� falder med tiden. S� i det lange l�b bliver de sammen. Men det
hele kunne v�re gjort billigere. 

En extension kunne v�re at l�gen ogs� havde en eller anden kvalitetsparamter.
S� fordi det bliver dyrer for patienten at skifte l�ge, s� kan l�gen
s�nke kvaliteten over tid, eller givet at l�gen har en eller anden
grad a alturisme �ge kvaliteten fordu de bliver billigere at f� signaler.}
\end{document}
